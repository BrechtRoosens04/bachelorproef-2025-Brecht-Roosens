%---------- Inleiding ---------------------------------------------------------

% TODO: Is dit voorstel gebaseerd op een paper van Research Methods die je
% vorig jaar hebt ingediend? Heb je daarbij eventueel samengewerkt met een
% andere student?
% Zo ja, haal dan de tekst hieronder uit commentaar en pas aan.

\paragraph{Opmerking}

Dit voorstel is gebaseerd op het onderzoeksvoorstel dat werd geschreven in het
kader van het vak Research Methods dat ik vorig academiejaar heb
uitgewerkt (met medestudent Liam Smets als mede-auteur).


\section{Inleiding}%
\label{sec:inleiding}

Graafmachines, hijskranen en betonpompen zijn cruciaal voor de dagelijkse activiteiten van kleine bouwbedrijven en hebben een directe impact op zowel de productiviteit als de kosten van een bouwproject waaraan het bedrijf werkt \autocite{Mikulic2024}. Als deze machines plotseling stuk raken, resulteert dit vaak in stilstand, vertragingen en aanzienlijke kosten voor reparaties van de machines \autocite{ErRatby2025}. In tal van kleine bouwondernemingen wordt onderhoud tegenwoordig nog altijd voornamelijk preventief uitgevoerd op vaste tijdstippen. Volgens het onderzoek van \textcite{Morganti2024} en \textcite{Deepak2025} tonen zij aan dat deze methode niet altijd effectief is en dit kan resulteren in onnodige onderhoudskosten of onvoorziene stilstanden van de machines.

Een effectievere optie is voorspellend onderhoud waarbij sensorgegevens en machine learnin\-g-algoritmes worden ingezet om problemen van tevoren te identificeren. Diverse onderzoeken wijzen erop dat voorspellend onderhoud kan leiden tot minder stilstand van de machines, maar ook zorgt voor lagere onderhoudskosten en grotere betrouwbaarheid van de machines zelf \autocite{Carvalho2019, ErRatby2025, Mahale2025}. Sensoren zoals trillingssensoren, temperatuursensoren en Inertial Measurement Units of IMU's worden vaak gebruikt om de toestand van machines in real-time te monitoren \autocite{Nayak2024, Molaei2024}. Volgens \textcite{Elahi2023} en \textcite{Wang2025} blijft de implementatie van deze technologie in de kleine bouwbedrijven echter beperkt door de factoren zoals ruis in de sensordata, beperkte datasets en een tekort aan technische kennis.

Deze bachelorproef is specifiek gebaseerd op kleine bouwbedrijven onder andere Bouwonderneming Marchand BV, waarmee ik in deze bachelorproef ga samenwerken, die ook een eigen machinepark heeft. De doelgroep van mijn onderzoek richt zich vooral op de IT-verantwoordelijk\-e van het bedrijf en de werknemers van het bedrijf waarvoor ik ga zorgen dat ze er ook meekunnen werken. De belangrijkste onderzoeksvraag is: Op welke manier kan een machine learning-model, gebaseeerd op sensorgegevens, worden gebruikt om storingen in bouwmachines bij een kleine bouwonderneming nauwkeurig te voorspellen en het onderhoud effectiever te organiseren? Eerst wordt onderzocht welke soorten sensordata en onderhoudsproblemen van belang zijn voor bouwmachines \autocite{Mikulic2024, Nayak2024}. Welke technische beperkingen er zijn binnen kleine bouwondernemingen \autocite{Elahi2023}. Zoals \textcite{Carvalho2019}, \textcite{Roehrich2024}, \textcite{Li2025} en \textcite{Wang2025} ook in hun onderzoek beschrijven, kan daarna pas onderzocht worden welke machine learning-algoritmes geschikt zijn voor voorspellend onderhoud en op welke manier deze correct kunnen worden beoordeeld.

Deze bachelorproef heeft als doel een soort van proof-of-concept voorspellend onderhoudsmodel te creëren en te evalueren voor de gekozen casus. Dit wordt ook aangevuld met een praktisch adviesrapport om het model te integreren en uit te voeren binnen het bedrijf. De bachelorproef is pas succesvol, zoals \textcite{Raffaele2024} en \textcite{Mahale2025} ook beschrijven in hun onderzoek, als het model technisch helemaal goed functioneert en een duidelijke waarde toevoegt aan het onderhoudsbeleid van het bouwbedrijf.

%---------- Stand van zaken ---------------------------------------------------

\section{Literatuurstudie}%
\label{sec:literatuurstudie}

\paragraph{Het onderhoud van bouwmachines en de huidige problemen}

Bouwmachines zijn essentieel voor de werking van bouwbedrijven en hun betrouwbaarheid heeft een directe impact op de productiviteit, planning en kosten\autocite{Mikulic2024}. Vandaag de dag wordt het onderhoud van deze machines voornamelijk preventief of reactief uitgevoerd, wat inhoudt dat het onderhoud op specifieke momenten of pas na een defect plaatsvindt\autocite{Mikulic2024, Morganti2024}. Verschillende onderzoeken van \textcite{ErRatby2025} en \textcite{Deepak2025} tonen aan dat deze methode niet effectief is, want soms worden onderdelen vervangen terwijl ze nog functioneren of onvoorziene defecten blijven optreden en resulteren in stilstand van de machine en brengt tevens ook hoge kosten met zich mee.

In de bouwsector wordt onderhoud nog ingewikkelder door zware omstandigheden, variabele belasting en externe factoren, wat resulteert in snellere slijtage en dat het voorspellen van een defect ook moeilijker wordt \autocite{Mikulic2024}. Hoewel datagebaseerde onderhoudsmethoden steeds vaker worden bestudeerd, blijft hun toepassing in kleine bouwbedrijven beperkt, omdat deze ondernemingen vaak niet over genoeg data, specifieke kennis of financiële middelen beschikken\autocite{Elahi2023}. Bovendien volgens \textcite{Wang2025} vormen ruis in de sensorgegevens en een slechte kwaliteit van die gegevens een aanzienlijke hindernis voor het opsporen van betrouwbare fouten in realistische situaties.

\paragraph{Voorspellend onderhoud met sensordata en machine learning-algoritmes}

Voorspellend onderhoud maakt gebruik van sensorgegevens en machine learning-algoritmes om problemen van tevoren te herkennen in plaats van pas na een probleem actie te ondernemen. Random Forests, Support Vector Machines en Neurale Netwerken zijn uiterst geschikt voor toepassing van voorspellend onderhoud \autocite{Carvalho2019}. Volgens recente onderzoeken van \textcite{Li2025} tonen zij aan dat deep learning-algoritmes ook uiterst precies zijn in het voorspellen van defecten op basis van sensorgegevens.

Er worden diverse soorten sensoren ingezet om betrouwbare inputdata te verzamelen. \textcite{Nayak2024} bewijzen dat trillingssensoren en temperatuurmetingen uiterst effectief zijn in het opsporen van aanvankelijke storingen in draaiende machines. In de bouwsector worden bovendien IMU-sensoren ingezet voor het analyseren van de bewegingen die de machine maakt, bijvoorbeeld door het automatisch herkennen van de werkprocessen die de graafmachines uitvoeren \autocite{Molaei2024}. Deze onderzoeken laten zien dat sensortechnologie op technisch vlak geschikt is voor gebruik in onder andere de industriële sector.

De voordelen van voorspellend onderhoud worden eveneens op bedrijfsniveau onderbouwd. \textcite{ErRatby2025} bewijzen dat voorspellend onderhoud resulteert in minder stilstanden van de machines, lagere onderhoudskosten en verbetering van de efficiëntie. Er blijven echter aanzienlijke obstakels bestaan met betrekking tot de betrouwbaarheid van de modellen, het classificeren van fouten en defecten en de implementatie in bestaande systemen \autocite{Roehrich2024, Elahi2023}. Volgens \textcite{Wang2025} en \textcite{Holtz2025} hebben nieuwe methoden, zoals stevige foutbeoordeling bij ruis in de data en data-efficiënte leermethoden, als doel deze problemen te verminderen, maar vragen nog om aanvullende praktische bevestiging in kleinere ondernemingen.

\paragraph{Conclusie en relevantie}

De literatuur toont aan dat voorspellend onderhoud met machine learni\-ng-algoritmes technisch goed onderbouwd is en al met succes wordt toegepast in de industriële sector \autocite{Carvalho2019, ErRatby2025, Li2025}.Verschillende onderzoeken laten zien dat het gebruik ervan in kleine bouwbedrijven eerder nog beperkt blijft door praktische en organisatorische beperkingen, zoals onder andere beperkte gegevens, de grote hoeveelheid ruis in de gegevens en de moeilijkheid van de implementatie \autocite{Elahi2023, Mikulic2024}. Dit vormt de basis voor deze bachelorproef waarin bestudeerd wordt op welke wijze een voorspellend onderhoudsmodel haalbaar kan worden ingezet binnen een klein bouwbedrijf. 

% Voor literatuurverwijzingen zijn er twee belangrijke commando's:
% \autocite{KEY} => (Auteur, jaartal) Gebruik dit als de naam van de auteur
%   geen onderdeel is van de zin.
% \textcite{KEY} => Auteur (jaartal)  Gebruik dit als de auteursnaam wel een
%   functie heeft in de zin (bv. ``Uit onderzoek door Doll & Hill (1954) bleek
%   ...'')

%---------- Methodologie ------------------------------------------------------
\section{Methodologie}%
\label{sec:methodologie}

Dit onderzoek maakt gebruik van een praktisch maar ook casusgerichte aanpak om zowel probleemgerichte als oplossingsgerichte deelvragen te beantwoorden. De methode past bij de huidige technieken voor voorspellend onderhoud, zoals gebruikt in door anderen eerdere uitgevoerd\-e case- en vergelijkende onderzoeken. Ik verwacht dat ik ongeveer 14 weken aan deze bachelorproef zal werken. Een paar weken zullen met elkaar overlappen, dit leg ik uit hieronder in de verschillende fases.

\paragraph{Literatuurstudie}

Eerst ga ik een literatuuronderzoek uitvoeren om verder het onderhoudsprobleem binnen de bouwsector beter te kunnen plaatsen en te achterhalen waarom gebruikelijke onderhoudsmethoden resulteren in stilstand en fouten, verder nog dit wordt al aangetoond in de industriële sector en de bouwmachines \autocite{ErRatby2025, Mikulic2024, Farwaha2024}. Bovendien wordt er ook onderzocht welke sensoren, datatypes en machine learning-algoritmes momenteel het beste functioneren voor het detecteren van fouten en het voorspellen van de resterende nuttige levensduur van een bepaald component van de machine \autocite{Carvalho2019, Nayak2024, Li2025, Meddaoui2024}. Met deze fase zal ik waarschijnlijk zo'n drie weken bezig zijn om dit af te werken en dit zal ook wel nog overlappen met andere fases als ik nog bepaalde bronnen nodig heb. Ik kan hiervan ook een deliverable maken die ik op het einde van die drie weken al eens kan indienen om mezelf een deadline op te leggen.

\paragraph{Analyse-fase}

Als een volgende fase ga ik de specifieke situatie bij Bouwonderneming Marcha\-nd Bv, waarmee ik nauw samenwerk in deze bachelorproef, bestuderen door middel van gesprekken met de technisch verantwoordelijke, geïnspireerd door de praktijkvoorbeelden in de onderhoudsonderzoeken van \textcite{Racchumi2024} en \textcite{Deepak2025}. De meest voorkomende storingen, onderhoudsintervallen en de belangrijkste sensoren worden hierbij vastgesteld. Met deze fase ga ik waarschijnlijk één tot misschien twee weken bezig zijn om het uitwerken. Dit kan ook overlapping hebben met de literatuurstudie.

\paragraph{Data-verzameling}

Ik kan de onderhouds-en fo\-utgegevens bij de machine-dealers gaan opvragen, aangezien kleine bouwondernemingen vaak geen eigen sensorgegevens registreren, maa\-r in voorspellend onderhoudsonderzoek is het gebruikelijk om gegevens te verzamelen via externe organisaties zoals de dealers \autocite{Elahi2023, Hu2024, Wang2025}. Als dat nodig is, kunnen er extra gesimuleerde of synthetische tijdreeksen worden ontwikkeld op basis van al geweten faalpatronen uit de literatuur \autocite{Nayak2024, Wang2025, Meddaoui2024}. Hiermee ga ik toch een viertal-weken bezig zijn om de dataset goed te kunnen vormen, bij de verschillende dealers langs gaan en misschien ook eventueel data te simuleren.

\paragraph{Preprocessing en feature engineering}

Ik ga in de deze fase de verzamelde gegevens weergeven, filteren en voorbereiden om te gebruiken voor een machine learning-model op te maken. Dit omvat onder andere het verwijderen van ruis in de dataset en het normaliseren van tijdreeksen, methoden die cruciaal zijn volgens de studies van \textcite{Wang2025} en \textcite{Aggarwal2025} over foutdiagnose onder realistische omstandigheden. Volgens \textcite{Nayak2024} en \textcite{Dang2024} is de feature engineering gebaseerd op de best practices van data van vibratiesensoren, belastingsensoren en multisensoren. Met deze fase ga ik één week werk aan hebben om de dataset goed te zetten om te gebruiken in de volgende fase.

\paragraph{Proof-of-concept machine learning-model}

Ve\-rvolgens ga ik verschillende machine learn\-ing-algoritmes ontwikkelen en uittesten. Onderzoek toont aan dat ensemble-methoden en deep learn\-ing-technieken een uitstekende prestatie leveren in voorspellingsmodellen in de industrie sector \autocite{Carvalho2019, Li2025, Keshavarz2025}. Hierdoor ga ik onder andere Random Forests, Gradient Boosting, CatBoost en CNN/LSTM-architecturen bestuderen en uitwerken \autocite{Khalili2025, Li2025, Dang2024}. De uitvoering vindt plaats in Python, met onder andere Pandas, Scikitlearn en PyTorch, meer nog \textcite{Chen2025} hebben onderzocht of multiview-time-seriesmod\-ellen die zijn gebruikt in het onderhoud van voertuigen, ook geschikt zijn voor bouwmachines. Ik ga met deze fase zo'n twee weken bezig zijn om het volledig uit te werken.

\paragraph{Beoordeling van het model}

Het model wordt in deze fase beoordeeld met gebruikelijke prestati\-e-indicatoren zoals precision, recall, F1-score en conf\-usion-matrices, want deze worden ook geadviseerd in de studies over foutclassificatie en onderhoudsdetectie \autocite{Roehrich2024, Mahale2025}. De uitkomsten worden in vergelijking gebracht met het bestaande preventieve onderhoud\-sbeleid van het bouwbedrijf. Volgens \textcite{Deepak2025} en \textcite{Hu2024} moet ik mij focussen op de meldingen die niet correct zijn, overbodige waarschuwingen en de relevantie in een realistische situatie voor het bouwbedrijf. Met deze fase ga ik één week bezig zijn om een goeie beoordeling te doen van mijn model.

\paragraph{conclusie en implementatie voor bouwbedrijf}

In deze laatste fase ga ik proberen de uitkomsten te interpreteren en probeer ik een technisch en praktisch adviesrapport op te stellen voor Bouwonderneming Marchand BV waarmee ik deze bachelorproef zal samenwerken. De invloed van het model is gebaseerd op literatuur die aantoont dat voorspellend onderhoud resulteert in lagere kosten, minder stilstand van de machines en een grotere betrouwbaarheid van deze machines\autocite{ErRatby2025, Mahfoud2025, Maguluri2024}. Het adviesrapport en het proof-of-concept vormen samen het defintieve resultaat van de bachelorproef die ik ook kan indienen als deliverables en gebruiken bij de mondelinge verdediging. Hiermee ga ik één week bezig zijn om uit te werken.

%---------- Verwachte resultaten ----------------------------------------------
\section{Verwacht resultaat, conclusie}%
\label{sec:verwachte_resultaten}

Aan de hand van de literatuur neem ik aan dat een machine learning-model bij Bouwonderneming Marchand BV in staat zal zijn om storingen van tevoren te voorspellen. Volgens de studies van \textcite{ErRatby2025} en \textcite{Mikulic2024} wijzen zij erop dat voorspellend onderhoud resulteert in minder stilstand van de machines en verbeterde prestaties in de industrie sector en dat dit ook een bekend probleem is bij bouwmachines.

Ik ben van mening dat sensorgegevens zoals trillingen en belasting genoeg informatie bieden om defecten te identificeren, omdat eerdere onderzoeken hebben aangetoond dat deze signalen geschikt zijn voor het opsporen van fouten wanneer ze goed worden gefilterd \autocite{Nayak2024, Wang2025}. In onderzoeken van \textcite{Carvalho2019} en \textcite{Li2025} hebben modellen als Random Forest, CatBoost en deep learning de beste prestatie geleverd, waardoor ik verwacht dat één van deze modellen heldere en nuttige resultaten zal opleveren.

Waarschijnlijk zal de uitkomst een precision-recall-grafiek zijn die laat zien hoeveel storingen nauwkeurig kunnen worden voorspeld, vergelijkbaar met eerdere voorbeelden van \textcite{Roehrich2024}. Volgens \textcite{Mahfoud2025} en \textcite{Deepak2025} kan ik hierdoor Bouwonderneming Marchand BV specifiek advies geven over welke datastromen ze moeten verzamelen en hoeveel efficiëntie er haalbaar is, iets waarbij kleine bouwbedrijven duidelijk voordeel kunnen halen.

\paragraph{conclusie}

Ik verwacht te concluderen dat voorspellend onderhoudsmodel technisch haalbaar is en een proof-of-conceptmodel nuttig kan zijn voor het verbeteren van het plannen van het onderhoud en het verminderen van de onvoorziene storingen en defecten.
