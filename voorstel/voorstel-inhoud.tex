%---------- Inleiding ---------------------------------------------------------

% TODO: Is dit voorstel gebaseerd op een paper van Research Methods die je
% vorig jaar hebt ingediend? Heb je daarbij eventueel samengewerkt met een
% andere student?
% Zo ja, haal dan de tekst hieronder uit commentaar en pas aan.

\paragraph{Opmerking}

Dit voorstel is gebaseerd op het onderzoeksvoorstel dat werd geschreven in het
kader van het vak Research Methods dat ik vorig academiejaar heb
uitgewerkt (met medestudent Liam Smets als mede-auteur).


\section{Inleiding}%
\label{sec:inleiding}

Graafmachines, hijskranen en betonpompen zijn cruciaal voor de dagelijkse activiteiten van kleine bouwbedrijven en hebben een directe impact op zowel de productiviteit als de kosten van een bouwproject waaraan het bedrijf werkt \autocite{Mikulic2024}. Als deze machines plotseling defect raken, resulteert dit vaak in stilstand, vertragingen en aanzienlijke kosten voor reparaties van de machines \autocite{ErRatby2025}. In tal van kleine bouwondernemingen wordt onderhoud tegenwoordig nog altijd voornamelijk preventief uitgevoerd op vaste tijdstippen. Volgens het onderzoek van \textcite{Morganti2024} en \textcite{Deepak2025} tonen zij aan dat deze methode niet altijd effectief is en dit kan resulteren in onnodige onderhoudskosten of onvoorziene stilstanden van de machines.

Een effectievere optie is voorspellend onderhoud waarbij sensorgegevens en machine learnin\-g-algoritmes worden ingezet om problemen van tevoren te identificeren. Diverse onderzoeken wijzen erop dat voorspellend onderhoud kan leiden tot minder stilstand van de machines, maar ook zorgt voor lagere onderhoudskosten en grotere betrouwbaarheid van de machines zelf \autocite{Carvalho2019, ErRatby2025, Mahale2025}. Sensoren zoals trillingssensoren, temperatuursensoren en Inertial Measurement Units of IMU's worden vaak gebruikt om de toestand van machines in real-time te monitoren \autocite{Nayak2024, Molaei2024}. Volgens \textcite{Elahi2023} en \textcite{Wang2025} blijft de implementatie van deze technologie in de kleine bouwbedrijven echter beperkt door de factoren zoals ruis in de sensordata, beperkte datasets en een tekort aan technische kennis.

Deze bachelorproef is specifiek gebaseerd op kleine bouwbedrijven onder andere Bouwonderneming Marchand BV, waarmee ik in deze bachelorproef ga samenwerken, die ook een eigen machinepark heeft. De doelgroep van mijn onderzoek richt zich vooral op de IT-verantwoordelijk\-e van het bedrijf en de werknemers van het bedrijf waarvoor ik ga zorgen dat ze er ook meekunnen werken. De belangrijkste onderzoeksvraag is: Op welke manier kan een machine learning-model, gebaseeerd op sensorgegevens, worden gebruikt om storingen in bouwmachines bij een kleine bouwonderneming nauwkeurig te voorspellen en het onderhoud effectiever te organiseren? 

De deelvragen van deze bachelorproef zijn ontwikkeld op basis van een analyse van de huidige literatuur over voorspellend onderhoud en het toepassen van machine learning-algoritmes in industriële en bouwkundige omgevingen. Diverse onderzoeken tonen aan dat traditionele onderhoudsmethoden in de bouwsector resulteren vaak in onnodige kosten voor onderhoud en onverwachte stilstanden , wat een nadelige impact heeft op de activiteiten van de kleine bouwbedrijven \autocite{Carvalho2019, Mikulic2024, ErRatby2025}. Eerst worden er enkele deelvragen geformuleerd binnen het probleemdomein op basis van deze conclusies. Deze deelvragen zijn gericht op het begrijpen van de beschikbare sensorgegevens, de gebruikelijke storingen en de beperkingen die bestaan in een realistische bouwomgeving.

\paragraph{Deelvragen in verband met het probleemdomein}

\begin{itemize}
    \item 1: Welke soorten sensordata zijn het meest geschikt voor het voorspellen van storingen in bouwmachines binnen kleine bouw ondernemingen?
    
    \item 2: Welke verschillen bestaan er in beschikbare sensordata tussen verschillende types bouwmachines en hoe beïnvloeden deze verschillen de mogelijkheid tot foutdetectie?
    
    \item 3: Welke typische storingen en onderhoudsproblemen komen het vaakst voor bij bouwmachines in kleine bouwondernemingen?
    
    \item 4: Welke vormen van ruis en onnauwkeurigheden komen voor in sensordata van bouwmachines en welke impact hebben deze op de betrouwbaarheid van foutdetectie?
    
    \item 5: Welke organisatorische, technische en financiële uitdagingen belemmeren de toepassing van het vorospellend onderhoud in kleine bouwondernemingen?
\end{itemize}

Deelvragen binnen het oplossingsdomein zijn gebaseerd op de inzichten uit het probleemdomein. Deze deelvragen richten zich op het ontwikkelen, implementeren en beoordelen van een voorspellend onderhoudssysteem. Op basis van machine learning-algoritmes, waarbij de haalbaarheid en praktische bruikbaarheid in overweging worden genomen.

\paragraph{Deelvragen in verband met het oplossingsdomein}

\begin{itemize}
    \item 1: Hoe kan sensordata van bouwmachines efficiënt worden voorverwerkt, gefilterd en geanalyseerd om geschikt te zijn voor machine learning-modellen?
    
    \item 2: Welke machine learning-algoritmes zijn het meest geschikt voor het ontwikkelen van een voorspellend onderhoudsmodel in een context met beperkte en ruisgevoelige data?
    
    \item 3: Hoe kunnen de geselecteerde machine learning-modellen worden getraind en gevalideerd om betrouwbare voorspellingen va\-n storingen te realiseren?
    
    \item 4: Wat is de invloed van verschillende feature selection- en feature engineering-methoden op de prestaties van het voorspellend onderhoudsmodel?
    
    \item 5: Welke evaluatiemaatstaven, zoals precision, recall, f1-score en confusion-matrix zijn het meest geschikt om de prestaties van het model te beoordelen in een onderhoudscontext?
    
    \item 6: Hoe kunnen valse positieven en foutieve voorspellingen worden geminimaliseerd om onnodige onderhoudsinterventies te vermijden?
    
    \item 7: Hoe kan de betrouwbaarheid en uitlegbaarheid van het voorspellend onderhoudsmodel worden verhoogd om praktische inzet in een bouwbedrijf mogelijk te maken?
    
    \item 8: Welke potentiële kostenbesparingen en efficiëntieverbeteringen kunnen worden gerealiseerd door de toepassing van het ontwikkelde voorspellend onderhoudsmodel?
\end{itemize}

Deze bachelorproef heeft als doel een soort van proof-of-concept voorspellend onderhoudsmodel te creëren en te evalueren voor de gekozen casus. Dit wordt ook aangevuld met een praktisch adviesrapport om het model te integreren en uit te voeren binnen het bedrijf. De bachelorproef is pas succesvol, zoals \textcite{Raffaele2024} en \textcite{Mahale2025} ook beschrijven in hun onderzoek, als het model technisch helemaal goed functioneert en een duidelijke waarde toevoegt aan het onderhoudsbeleid van het bouwbedrijf.

%---------- Stand van zaken ---------------------------------------------------

\section{Literatuurstudie}%
\label{sec:literatuurstudie}

\paragraph{Het onderhoud van bouwmachines en de huidige problemen}

Bouwmachines zijn essentieel voor de werking van bouwbedrijven en hun betrouwbaarheid heeft een directe impact op de productiviteit, planning en kosten\autocite{Mikulic2024}. Vandaag de dag wordt het onderhoud van deze machines voornamelijk preventief of reactief uitgevoerd, wat inhoudt dat het onderhoud op specifieke momenten of pas na een defect plaatsvindt\autocite{Mikulic2024, Morganti2024}. Verschillende onderzoeken van \textcite{ErRatby2025} en \textcite{Deepak2025} tonen aan dat deze methode niet effectief is, want soms worden onderdelen vervangen terwijl ze nog functioneren of onvoorziene defecten blijven optreden en resulteren in stilstand van de machine en brengt tevens ook hoge kosten met zich mee.

In de bouwsector wordt onderhoud nog ingewikkelder door zware omstandigheden, variabele belasting en externe factoren, wat resulteert in snellere slijtage en dat het voorspellen van een defect ook moeilijker wordt \autocite{Mikulic2024}. Hoewel datagebaseerde onderhoudsmethoden steeds vaker worden bestudeerd, blijft hun toepassing in kleine bouwbedrijven beperkt, omdat deze ondernemingen vaak niet over genoeg data, specifieke kennis of financiële middelen beschikken\autocite{Elahi2023}. Bovendien volgens \textcite{Wang2025} vormen ruis in de sensorgegevens en een slechte kwaliteit van die gegevens een aanzienlijke hindernis voor het opsporen van betrouwbare fouten in realistische situaties.

\paragraph{Voorspellend onderhoud met sensordata en machine learning-algoritmes}

Voorspellend onderhoud maakt gebruik van sensorgegevens en machine learning-algoritmes om problemen van tevoren te herkennen in plaats van pas na een probleem actie te ondernemen. Random Forests, Support Vector Machines en Neurale Netwerken zijn uiterst geschikt voor toepassing van voorspellend onderhoud \autocite{Carvalho2019}. Volgens recente onderzoeken van \textcite{Li2025} tonen zij aan dat deep learning-algoritmes ook uiterst precies zijn in het voorspellen van defecten op basis van sensorgegevens.

Er worden diverse soorten sensoren ingezet om betrouwbare inputdata te verzamelen. \textcite{Nayak2024} bewijzen dat trillingssensoren en temperatuurmetingen uiterst effectief zijn in het opsporen van aanvankelijke storingen in draaiende machines. In de bouwsector worden bovendien IMU-sensoren ingezet voor het analyseren van de bewegingen die de machine maakt, bijvoorbeeld door het automatisch herkennen van de werkprocessen die de graafmachines uitvoeren \autocite{Molaei2024}. Deze onderzoeken laten zien dat sensortechnologie op technisch vlak geschikt is voor gebruik in onder andere de industriële sector.

De voordelen van voorspellend onderhoud worden eveneens op bedrijfsniveau onderbouwd. \textcite{ErRatby2025} bewijzen dat voorspellend onderhoud resulteert in minder stilstanden van de machines, lagere onderhoudskosten en verbetering van de efficiëntie. Er blijven echter aanzienlijke obstakels bestaan met betrekking tot de betrouwbaarheid van de modellen, het classificeren van fouten en defecten en de implementatie in bestaande systemen \autocite{Roehrich2024, Elahi2023}. Volgens \textcite{Wang2025} en \textcite{Holtz2025} hebben nieuwe methoden, zoals stevige foutbeoordeling bij ruis in de data en data-efficiënte leermethoden, als doel deze problemen te verminderen, maar vragen nog om aanvullende praktische bevestiging in kleinere ondernemingen.

In recente literatuur worden naast supervised machine learning-ritmes ook unsupervised en self-supervised leermethoden gebruikt voor voorspellend onderhoud, vooral wanneer gelabelde foutdata beperkt beschikbaar zijn \autocite{Elahi2023}. \textcite{Giannoulidis2024} beschrijven verschillende unsupervised benaderingen, waaronder similartiy-based methoden die afwijkingen detecteren door te vergelijken met de patronen van normaal machinegedrag, forecasting-based modellen die anomalieën identificeren op basis van grote voorspellingsfouten in tijdsreeksen en deep unsupervised modellen zoals autoencoders die afwijkingen vaststellen via reconstructiefouten.

Bovendien worden er ook self-supervised deep learning-modellen toegepast. \textcite{Aggarwal2025} stellen met DASTAD een Transformer-gebaseerd self-supervised model voor anomaly detection in multivariate tijdsreeksen.\textcite{Holtz2025} combineren unsupervised anomaly detection op basis van autoencoders met active learning om met een beperkte hoeveelheid gelabelde data toch betrouwbare foutdetectie te realiseren.

\paragraph{Conclusie en relevantie}

De literatuur toont aan dat voorspellend onderhoud met machine learnin\-g resulteert in minder onvoorziene stilstand en een efficiënter onderhoud in industriële omgevingen \autocite{Carvalho2019, ErRatby2025}. Voor bouwmachines wordt onderhoud nog steeds vaak op een preventieve of reactieve manier uitgevoerd, wat inefficiëntie en moeilijk te voorspellen defecten met zich meebrengt \autocite{Mikulic2024}. Volgens verschillende onderzoeken van \textcite{Elahi2023} en \textcite{Wang2025} wijzen erop dat supervised learning in deze context slechts gedeeltelijk toepasbaar is, omdat er geen gelabelde fouten zijn en er ruis is in de sensordata.

Recent onderzoek laat zien dat zowel unsupervised als self-supervised learning effectiever zijn voor onderhoudstoepassingen die werken met beperkte en onvolledige gegevens, onder andere door middel van anomaly detection, autoencoders en modellen die gebaseerd zijn op transformers \autocite{Giannoulidis2024, Aggarwal2025, Holtz2025}. Deze benaderingen zijn stevig in dynamische en onvoorspelbare omgevingen, waardoor ze van groot belang zijn voor de bouwsector \autocite{Mahale2025}. Een nadruk op unsupervised en self-supervised learni\-ng lijkt meer geschikt en haalbaar alternatief voor voorspellend onderhoud binnen kleine bouwbedrijven.

% Voor literatuurverwijzingen zijn er twee belangrijke commando's:
% \autocite{KEY} => (Auteur, jaartal) Gebruik dit als de naam van de auteur
%   geen onderdeel is van de zin.
% \textcite{KEY} => Auteur (jaartal)  Gebruik dit als de auteursnaam wel een
%   functie heeft in de zin (bv. ``Uit onderzoek door Doll & Hill (1954) bleek
%   ...'')

%---------- Methodologie ------------------------------------------------------
\section{Methodologie}%
\label{sec:methodologie}

Dit onderzoek hanteert een praktische en casusgerichte benadering om zowel probleemgerichte als oplossingsgerichte vragen te beantwoorden. De aanpak past bij de bestaande technieken voor voorspellend onderhoud, zoals gebruikt in eerdere case-en vergelijkende studies van anderen. Deze bachelorproef zal ongeveer veertien weken in beslag nemen. Enkele weken zullen elkaar overlappen, dit wordt hieronder verder uitgelegd in de verschillende fases.

\paragraph{Literatuurstudie}

In de eerste plaats wordt er een literatuurstudie uitgevoerd om het onderhoudsprobleem in de bouwsector beter te kunnen plaatsen en te onderzoeken waarom gangbare onderhoudsmethoden leiden tot stilstand en fouten. Dit is al bewezen in de industrie en bij bouwmachines \autocite{ErRatby2025, Farwaha2024, Mikulic2024}. Bovendien wordt onderzocht welke sensoren, datatypes en machine learning-algoritmes momenteel het me\-est effectief zijn voor het opsporen van fouten en het voorspellen van de resterende levensduur van bepaalde machinecomponenten \autocite{Carvalho2019, Li2025, Meddaoui2024, Nayak2024}. Deze fase wordt verwacht ongeveer drie weken te duren en er kan mogelijk een overlap ontstaan met andere fases als er extra bronnen vereist zijn. Daarnaast kan er een deliverable worden opgesteld dat aan het einde van deze drie weken ingediend kan worden om een heldere deadline te stellen.

\paragraph{Analyse-fase}

In deze fase wordt de specifieke situatie bij Bouwonderneming Marchand BV onderzocht, in samenwerking met het bedrijf, door middel van gesprekken met de technische verantwoordelijke. Deze methode is gebaseerd op de praktijkvoorbeelden uit de onderhoudsonderzoeken van \textcite{Racchumi2024} en \textcite{Deepak2025}. De meest voorkomende problemen, onderhoudsintervallen en de belangrijkste sensoren worden hierbij bepaald. Het uitwerken van deze fase zal naar verwachting één tot twee weken in beslag nemen, met mogelijke overlap met de literatuurstudie.

\paragraph{Data-verzameling}

Onderhouds- en foutgegevens kan worden opgevraagd bij de machine-dealers, omdat kleine bouwbedrijven vaak geen eigen sensorgegevens registreren. In de studies van \textcite{Elahi2023}, \textcite{Hu2024} en \textcite{Wang2025} over voorspellend onderhoud is het gebruikelijk om informatie te verkrijgen van externe organisaties waaronder dealers van de machines. Als dat nodig is, kunnen er extra gesimuleerde of synthetische tijdreeksen of data worden gecreëerd op basis van eerder bekende faalpatronen uit de literatuur \autocite{Nayak2024, Wang2025, Meddaoui2024}. Er bestaat een kans dat sommige dealers geen toegang willen geven tot hun data, daarop heb ik een alternatieve aanpak uitgewerkt in een volgende sectie van de methodologie namelijk de sectie `Risico en alternatieve aanpak` gebaseerd op synthetische data. Voor het opstellen van een complete dataset wordt ongeveer vier weken besteed, inclusief van het bezoeken of contacteren van de diverse dealers en indien nodig het simuleren van extra bijkomende data. 

Er is ondertussen voor het bachelorproefperiode begint al contact gelegd met de verantwoordelijke van Bouwonderneming Marchand BV en ook al contacten uitgegooid naar een aantal dealers van bouwmachines zoals onder andere Liebherr, het merk van één vande hijskranen in het bedrijf.

\paragraph{Preprocessing en feature engineering}

In deze fase worden de verzamelde gegevens weergegeven, gefilterd en klaargemaakt voor gebruik in een machine learning-model. Dit omvat onder andere het verwijderen van ruis uit de dataset en het normaliseren van tijdsreeksen, methoden die volgens de onderzoeken van \textcite{Wang2025} en \textcite{Aggarwal2025} essentieel worden beschouwd voor foutdiagnose onder realistische situaties. \textcite{Nayak2024} en \textcite{Dang2024} stellen dat de feature engineering gebaseerd is op de best practices voor gegevens die afkomstig zijn van vibratiesensoren, belastingsensoren en multisensoren. Voor deze fase is er ongeveer een week nodig om de dataset klaar te maken voor de volgende fase.

\paragraph{Proof-of-concept machine learning-model}

Er worden diverse machine learning algoritmes ontwikkeld en beoordeeld om onderhoudsbehoeften te voorspellen op basis van sensorgegevens. Volgens \textcite{Elahi2023}, \textcite{Giannoulidis2024} en \textcite{Aggarwal2025} is gelabelde foutdata bij kleine bouwbedrijven vaak beperkt en richt men zich voornamelijk op unsupervised en self-supervised technieken, die het mogelijk maken om afwijkingen in sensorgegevens te identificeren zonder duidelijke labels. 

De bestudeerde algoritmen omvatten similarit\-y-based methoden zoals Cosinus‑similariteit, fo\-recasting-modellen voor tijdsreeksen zoals LSTM, deep unsupervised modellen zoals autoencoders en een self-supervised Transformer (DASTAD) voor anomaly detection in multivariate sensordata \autocite{Giannoulidis2024, Aggarwal2025}. Volgens \textcite{Carvalho2019}, \textcite{Li2025} en \textcite{Khalili2025} bieden verschillende supervised modellen, zoals Random Forests, Gradient Boosting en CatBoost ook goed resultaten bij voldoende gelabelde gegevens. Deze worden ook bestudeerd en beoordeeld.

De uitvoering vindt plaats in Python met Pandas, Scikit-learn en PyTorch en er wordt gekeken in welke mate tijdreeksmodellen uit andere domeinen, zoals multiview time-seriesmodellen voor voertuigen, relevant zijn voor bouwmachines \autocite{Chen2025} Voor deze fase zijn er ongeveer twee weken gepland.

\paragraph{Beoordeling van het model}

In deze fase wordt het model geëvalueerd met gebruikelijke prestatie-indicatoren zoals precision, recall, F1-score en conf\-usion-matrices zoals aanbevolen in de studies over foutclassificatie en onderhoudsdetectie \autocite{Roehrich2024, Mahale2025}. De resultaten worden vergeleken met het bestaande preventieve onderhoudsbeleid van het bouwbedrijf. Volgens \textcite{Deepak2025} en \textcite{Hu2024} richt men zich op onjuiste meldingen, overbodige waarschuwingen en de relevantie van de resultaten in een realistische situatie voor het bouwbedrijf. Voor deze fase wordt ongeveer een week beschikbaar gesteld om een grondige evaluatie van het model te maken.

\paragraph{Risico en alternatieve aanpak}

Als de betrokken dealers van de machines die Bouwonderneming Marchand Bv bezit geen toegang willen of kunnen geven tot genoeg historische sensordata, wordt er als alternatief gebruikgemaakt van synthetische data. Volgens \textcite{Elahi2023} en \textcite{Carvalho2019} worden synthetische gegevens vaak toegepast wanneer real-world gegevens beperkt, onvolledig of vertrouwelijk zijn, zeker binnen industriële omgevingen zoals voorspellend onderhoud. Door realistische sensorgegevens te imiteren op basis van bekende gedragingen van machines, kan er toch een proof-of-concept worden ontwikkeld.

Volgens onderzoek vertonen modellen die alleen zijn getraind met synthetische gegevens beperkingen hebben wanneer ze worden gebruikt op echte machines, omdat gesimuleerde gegevens niet alle ruis, degradatiepatronen en onverwachte variaties uit de werkelijkheid bevatten \autocite{Meddaoui2024, Giannoulidis2024}. Volgens \textcite{Carvalho2019} worden synthetische data in de literatuur voornamelijk beschouwd als geschikt voor het valideren van methodologieën, het vergelijken van modellen en het evalueren van de algoritmische haalbaarheid in plaats van voor directe operationele toepassingen.

Recent onderzoek in voorspellend onderhoud laat zien dat unsupervised en self-supervised lear\-ning minder afhankelijk zijn van specifieke labeling, wat resulteert in een betere generalisatie van gesimuleerde gegevens naar echte gegevens, vo\-oral wanneer het doel anomaliedetectie is \autocite{Aggarwal2025, Holtz2025}. Volgens de onderzoeken van \textcite{Hu2024} en \textcite{Maguluri2024} worden digitale tweelingen en gesimuleerde omgevingen bovendien steeds vaker ingezet voor het genereren van synthetische data die nauwkeurig overeenkomen met de werkelijke industriële processen.

In deze bachelorproef zouden synthetische dat\-a worden gebruikt als alternatief plan om de technische uitvoerbaarheid van voorspellend onderhoud te bewijzen. Volgens \textcite{ErRatby2025} en \textcite{Raffaele2024} ligt de focus dan niet op absolute voorspellingsnauwkeurigheid, m\-aar op het evalueren van geschikte algoritmen, datastromen en evaluatiemethoden voor bouwmachines, wat volgens eerdere studies een aanvaardbare en wetenschappelijk onderbouwde aanpak is in een proof-of-concept context.

\paragraph{conclusie en implementatie voor bouwbedrijf}

In deze laatste fase worden de resultaten beoordeeld en wordt er een technisch en praktisch adviesrapport opgesteld voor Bouwonderneming M\-archand BV, het bedrijf waarmee deze bachelorproef wordt uitgevoerd. De impact van het model is gebaseerd op literatuur die aantoont dat voorspellend onderhoud resulteert in lagere kosten, minder stilstand van machines en een hogere betrouwbaarheid van deze machines \autocite{ErRatby2025, Mahfoud2025, Maguluri2024}. Het adviesrapport en het proof-of-concept vormen samen het definitieve resultaat van de bachelorproef ne kunnen als deliverables worden  ingediend en gebruikt worden tijdens de mondelinge verdediging. Voor deze fase wordt ongeveer één week vrijgemaakt om de uitwerking te voltooien.

%---------- Verwachte resultaten ----------------------------------------------
\section{Verwacht resultaat, conclusie}%
\label{sec:verwachte_resultaten}

Aan de hand van de huidige literatuur wordt er verwacht dat een machine-learningmodel binnen Bouwonderneming Marchand BV kan helpen om storingen in bouwmachines vroegtijdig te voorspellen. Volgens onderzoek van \textcite{ErRatby2025} en \textcite{Mikulic2024} resulteert voorspellend onderhoud in minder onvoorziene stilstand en een efficiëntere werking van machines, terwijl het onderhoud van bouwmachines tegenwoordig nog steeds vaak op een preventieve of reactieve wijze plaatsvindt.

Volgens onderzoek zijn sensorgegevens zoals trillingen en belasting effectief voor het opsporen van beginnende defecten, mits deze gegevens correct worden verwerkt en geanalyseerd \autocite{Nayak2024, Wang2025}. Aangezien gelabelde fouten in de praktijk vaak beperkt zijn, vooral bij kleine bedrijven, toont recent onderzoek van \textcite{Giannoulidis2024}, \textcite{Aggarwal2025} en \textcite{Holtz2025} aan dat zowel unsupervised als self-supervised learning effectiever zijn voor voorspellend onderhoud in realistische industriële omgevingen.

Als er genoeg historische gegevens beschikbaar zijn, laten vergelijkende onderzoeken zien dat machine learning-algoritmes zoals Random Forest, CatBoost en deep learning effectief zijn in predictieve onderhoudstoepassingen \autocite{Carvalho2019, Li2025}. Volgens \textcite{Roehrich2024} kunnen de uitkomsten van het model worden beoordeeld met objectieve criteria zoals precision en recall, bijvoorbeeld door middel van een precision-recall-grafiek, een gebruikelijke methode voor evaluatie bij foutdetectie.

Tot slot word er verwacht dat een proof-of-concept voor voorspellend onderhoud, met een nadruk op unsupervised en self-supervised learning. Volgens \textcite{Mahfoud2025} en \textcite{Deepak2025} kan dit resulteren in een verbeterd onderhoudsbeheer en minder onvoorziene machinestilstanden voor Bouwonderneming Marchand BV, dit wordt ook in industriële omgevingen aangetoond.
