%---------- Inleiding ---------------------------------------------------------

% TODO: Is dit voorstel gebaseerd op een paper van Research Methods die je
% vorig jaar hebt ingediend? Heb je daarbij eventueel samengewerkt met een
% andere student?
% Zo ja, haal dan de tekst hieronder uit commentaar en pas aan.

\paragraph{Opmerking}

Dit voorstel is gebaseerd op het onderzoeksvoorstel dat werd geschreven in het
kader van het vak Research Methods dat ik vorig academiejaar heb
uitgewerkt (met medestudent Liam Smets als mede-auteur).


\section{Inleiding}%
\label{sec:inleiding}

Graafmachines, hijskranen en betonpompen zijn cruciaal voor de dagelijkse activiteiten van kleine bouwbedrijven en hebben een directe impact op zowel de productiviteit als de kosten van een bouwproject waaraan het bedrijf werkt \autocite{Mikulic2024}. Als deze machines plotseling stuk raken , resulteert dit vaak in stilstand, vertragingen en aanzienlijke kosten voor reparaties van de machines \autocite{ErRatby2025}. In tal van kleine bouwondernemingen wordt onderhoud tegenwoordig nog altijd voornamelijk preventief uitgevoerd op vaste tijdstippen. Volgens het onderzoek van \textcite{Morganti2024} en \textcite{Deepak2025} tonen zij aan dat deze methode niet altijd effectief is en dit kan resulteren in onnodige onderhoudskosten of onvoorziene stilstanden van de machines.

Een effectievere optie is voorspellend onderhoud waarbij sensorgegevens en machine learnin\-g-algoritmes worden ingezet om problemen van tevoren te identificeren. Diverse onderzoeken wijzen erop dat voorspellend onderhoud kan leiden tot minder stilstand van de machines, maar ook zorgt voor lagere onderhoudskosten en grotere betrouwbaarheid van de machines zelf \autocite{Carvalho2019, ErRatby2025, Mahale2025}. Sensoren zoals trillingssensoren, temperatuursensoren en Inertial Measurement Units of IMU's worden vaak gebruikt om de toestand van machines in real-time te monitoren \autocite{Nayak2024, Molaei2024}. Volgens \textcite{Elahi2023} en \textcite{Wang2025} blijft de implementatie van deze technologie in de kleine bouwbedrijven echter beperkt door de factoren zoals ruis in de sensordata, beperkte datasets en een tekort aan technische kennis.

Deze bachelorproef is specifiek gebaseerd op kleine bouwbedrijven onder andere Bouwonderneming Marchand BV ,waarmee ik in deze bachelorproef ga samenwerken, die ook een eigen machinepark heeft. De doelgroep van mijn onderzoek richt zich vooral op de IT-verantwoordelijk\-e van het bedrijf en de werknemers van het bedrijf waarvoor ik ga zorgen dat ze er ook meekunnen werken. De belangrijkste onderzoeksvraag is: Op welke manier kan een machine learning-model, gebaseeerd op sensorgegevens, worden gebruikt om storingen in bouwmachines bij een kleine bouwonderneming nauwkeurig te voorspellen en het onderhoud effectiever te organiseren? Eerst wordt onderzocht welke soorten sensordata en onderhoudsproblemen van belang zijn voor bouwmachines \autocite{Mikulic2024, Nayak2024}. Welke technische beperkingen er zijn binnen kleine bouwondernemingen \autocite{Elahi2023}. Zoals \textcite{Carvalho2019}, \textcite{Roehrich2024}, \textcite{Li2025} en \textcite{Wang2025} ook in hun onderzoek beschrijven, kan daarna pas onderzocht worden welke machine learning-algoritmes geschikt zijn voor voorspellend onderhoud en op welke manier deze correct kunnen worden beoordeeld.

Deze bachelorproef heeft als doel een soort van proof-of-concept voorspellend onderhoudsmodel te creëren en te evalueren voor de gekozen casus. Dit wordt ook aangevuld met een praktisch adviesrapport om het model te integreren en uit te voeren binnen het bedrijf. De bachelorproef is pas succesvol, zoals \textcite{Raffaele2024} en \textcite{Mahale2025} ook beschrijven in hun onderzoek, als het model technisch helemaal goed functioneert en een duidelijke waarde toevoegt aan het onderhoudsbeleid van het bouwbedrijf.

%---------- Stand van zaken ---------------------------------------------------

\section{Literatuurstudie}%
\label{sec:literatuurstudie}

\paragraph{Het onderhoud van bouwmachines en de huidige problemen}

Bouwmachines zijn essentieel voor de werking van bouwbedrijven en hun betrouwbaarheid heeft een directe impact op de productiviteit, planning en kosten\autocite{Mikulic2024}. Vandaag de dag wordt het onderhoud van deze machines voornamelijk preventief of reactief uitgevoerd, wat inhoudt dat het onderhoud op specifieke momenten of pas na een defect plaatsvindt\autocite{Mikulic2024, Morganti2024}. Verschillende onderzoeken van \textcite{ErRatby2025} en \textcite{Deepak2025} tonen aan dat deze methode niet effectief is, want soms worden onderdelen vervangen terwijl ze nog functioneren of onvoorziene defecten blijven optreden en resulteren in stilstand van de machine en brengt tevens ook hoge kosten met zich mee.

In de bouwsector wordt onderhoud nog ingewikkelder door zware omstandigheden, variabele belasting en externe factoren, wat resulteert in snellere slijtage en dat het voorspellen van een defect ook moeilijker wordt \autocite{Mikulic2024}. Hoewel datagebaseerde onderhoudsmethoden steeds vaker worden bestudeerd, blijft hun toepassing in kleine bouwbedrijven beperkt, omdat deze ondernemingen vaak niet over genoeg data, specifieke kennis of financiële middelen beschikken\autocite{Elahi2023}. Bovendien volgens \textcite{Wang2025} vormen ruis in de sensorgegevens en een slechte kwaliteit van die gegevens een aanzienlijke hindernis voor het opsporen van betrouwbare fouten in realistische situaties.

\paragraph{Voorspellend onderhoud met sensordata en machine learning-algoritmes}

% Voor literatuurverwijzingen zijn er twee belangrijke commando's:
% \autocite{KEY} => (Auteur, jaartal) Gebruik dit als de naam van de auteur
%   geen onderdeel is van de zin.
% \textcite{KEY} => Auteur (jaartal)  Gebruik dit als de auteursnaam wel een
%   functie heeft in de zin (bv. ``Uit onderzoek door Doll & Hill (1954) bleek
%   ...'')

%---------- Methodologie ------------------------------------------------------
\section{Methodologie}%
\label{sec:methodologie}

Hier beschrijf je hoe je van plan bent het onderzoek te voeren. Welke onderzoekstechniek ga je toepassen om elk van je onderzoeksvragen te beantwoorden? Gebruik je hiervoor literatuurstudie, interviews met belanghebbenden (bv.~voor requirements-analyse), experimenten, simulaties, vergelijkende studie, risico-analyse, PoC, \ldots?

Valt je onderwerp onder één van de typische soorten bachelorproeven die besproken zijn in de lessen Research Methods (bv.\ vergelijkende studie of risico-analyse)? Zorg er dan ook voor dat we duidelijk de verschillende stappen terug vinden die we verwachten in dit soort onderzoek!

Vermijd onderzoekstechnieken die geen objectieve, meetbare resultaten kunnen opleveren. Enquêtes, bijvoorbeeld, zijn voor een bachelorproef informatica meestal \textbf{niet geschikt}. De antwoorden zijn eerder meningen dan feiten en in de praktijk blijkt het ook bijzonder moeilijk om voldoende respondenten te vinden. Studenten die een enquête willen voeren, hebben meestal ook geen goede definitie van de populatie, waardoor ook niet kan aangetoond worden dat eventuele resultaten representatief zijn.

Uit dit onderdeel moet duidelijk naar voor komen dat je bachelorproef ook technisch voldoen\-de diepgang zal bevatten. Het zou niet kloppen als een bachelorproef informatica ook door bv.\ een student marketing zou kunnen uitgevoerd worden.

Je beschrijft ook al welke tools (hardware, software, diensten, \ldots) je denkt hiervoor te gebruiken of te ontwikkelen.

Probeer ook een tijdschatting te maken. Hoe lang zal je met elke fase van je onderzoek bezig zijn en wat zijn de concrete \emph{deliverables} in elke fase?

%---------- Verwachte resultaten ----------------------------------------------
\section{Verwacht resultaat, conclusie}%
\label{sec:verwachte_resultaten}

Hier beschrijf je welke resultaten je verwacht. Als je metingen en simulaties uitvoert, kan je hier al mock-ups maken van de grafieken samen met de verwachte conclusies. Benoem zeker al je assen en de onderdelen van de grafiek die je gaat gebruiken. Dit zorgt ervoor dat je concreet weet welk soort data je moet verzamelen en hoe je die moet meten.

Wat heeft de doelgroep van je onderzoek aan het resultaat? Op welke manier zorgt jouw bachelorproef voor een meerwaarde?

Hier beschrijf je wat je verwacht uit je onderzoek, met de motivatie waarom. Het is \textbf{niet} erg indien uit je onderzoek andere resultaten en conclusies vloeien dan dat je hier beschrijft: het is dan juist interessant om te onderzoeken waarom jouw hypothesen niet overeenkomen met de resultaten.

